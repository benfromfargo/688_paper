\documentclass{beamer}
\usetheme{Boadilla}

\title{The Causes of Preemption in the U.S. States}
\author{Ben Goehring}

\date{November 22, 2019}

\begin{document}

\begin{frame}
\titlepage
\end{frame}


\begin{frame}
\frametitle{Background and Research Question}

\begin{itemize}
\item States in the U.S. can preempt local ordinances by passing legislation preventing municipal regulations in certain policy domains.
\begin{itemize}
	\item Between 2013 and 2017, 14 states banned municipalities from raising the minimum wage above the statewide wage and 17 states prevented municipalities from mandating employers provide some form of paid sick or family medical leave.
	\item In the 1980s and 1990s, states preempted numerous municipal gun and tobacco regulations.
\end{itemize}
\item When and why do states preempt local ordinances? 
\begin{itemize}
	\item Extant literature emphasizes the role of conservative ideology, interest groups, and Republican control of government, but existing studies largely focus on preemption occurring after 2010 and do not control for cities' policymaking.
\end{itemize}
\end{itemize}
\end{frame}

\begin{frame}
\frametitle{Hypotheses}
Dependent variable: The probability of a state preempting a local ordinance in a given year. 

A state is most likely to preempt a specific municipal regulation when the following three conditions hold:
\begin{enumerate}
	\item At least one city in the state is able to pass, or has already passed, the regulation.
	\begin{itemize}
		\item Legal status of cities in the state 
		\item Whether a city in the state (or nearby state) has passed the relevant regulation 
	\end{itemize}
	\item There is a wide divergence between the policy preferences of the state and at least one city in the state.
	\begin{itemize}
		\item Relative liberalism of cities in the state
		\item Republican control of state government
	\end{itemize}
	\item Relevant interest groups are organized at the state level.
	\begin{itemize}
		\item Strength of relevant interest groups like ALEC, the Chamber of Commerce, and Tobacco lobbying firms
	\end{itemize}
\end{enumerate}
\end{frame}

\begin{frame}
\frametitle{Data: Tobacco and the Minimum Wage}
Restricting myself to a small number of policy areas keeps the data collection process tractable, as I need to collect data on both state preemption laws and cities' ordinances.
\begin{itemize}
	\item Regulating tobacco
	\begin{itemize}
		\item American Nonsmokers' Rights Foundation's U.S. Tobacco Control Laws Database contains data on all states' tobacco-related preemption laws and all local tobacco regulations. 
	\end{itemize}
	\item Raising the minimum wage
	\begin{itemize}
		\item Policy organizations track states' preemption policies and localities' minimum wages, but I need to collect data on local policies in states that now preempt municipalities from raising the minimum wage. 
	\end{itemize}
\end{itemize}
\end{frame}

\end{document}
